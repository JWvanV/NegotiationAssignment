\documentclass[11pt,a4paper]{report}
\usepackage[utf8]{inputenc}
\usepackage{amsmath}
\usepackage{amsfonts}
\usepackage{amssymb}
\usepackage{graphicx}
\usepackage{soul}
\usepackage{fullpage}
\usepackage{float}
\restylefloat{table}

\begin{document}

\begin{titlepage}
\begin{center}

% Upper part of the page. The '~' is needed because \\
% only works if a paragraph has started.


\textsc{\LARGE Delft University of Technology}\\[1.5cm]

\textsc{\Large IN4010 Practical Assignment 2}\\[0.5cm]

% Title
%\HRule \\[0.4cm]
{ \huge \bfseries Automated Negotiation \\[0.4cm] }

%\HRule \\[1.5cm]

% Author and supervisor
\noindent
\begin{minipage}{0.4\textwidth}
\begin{flushleft} \large
\emph{Group 11:}\\
Hidde \textsc{Coehoorn}\\
Ralf \textsc{Nieuwenhuizen}\\
Jan-Willem \textsc{van Velzen}
\end{flushleft}
\end{minipage}%
\begin{minipage}{0.4\textwidth}
\begin{flushright} \large
\emph{Supervisor:} \\
Reyhan \textsc{Aydogan}
\end{flushright}
\end{minipage}

\vfill

% Bottom of the page
{\large \today}

\end{center}
\end{titlepage}

\newpage

10 A4 pages may be enough, 15 A4 pages maximum\\







\section*{Introduction to assignment}

As part of the TU Delft course IN4010 Artificial Intelligence Techniques completion of a practical assignment on automated negotiation is required. This is the assignment report of group 11, of which the group members are Hidde Coehoorn, Ralf Nieuwenhuizen and Jan-Willem van Velzen.\\

The subject of the assignment is multilateral negotiation taking place in a multi-issue negotiation domain with discrete issues. The challenge is to design and implement a negotiating agent in GENIUS which can guide the negotiation process and help the parties involved with reaching a satisfying agreement. Various scenarios are created to test the agent, ranging in the degree of conflict between parties.\\

\textbf{Domain}\\
Each group had to define their own negotiation domain, complete with multiple issues. Our group chose to create the 'Purchasing a Car' domain. This domain has a total of five issues, each with a number of issue values ranging from two to four (see Table \ref{tab:domainissues}).

\begin{table}[H]
\centering
\caption{The domain issues and issue values.}
\label{tab:domainissues}
\begin{tabular}{|c|c|c|c|c|}
\hline
Power (hp) & Capacity & Mileage (km) & Fuel consumption (km/l) & Wheels \\
\hline
less than 200 & 2 & Below 10.000 & 5 & 3 \\
200 or more & 5 & Between 10.000 and 20.000 & 10 & 4 \\
& 7 & Above 20.000 & 15 & 6 \\
& & & 20 & \\ 
\hline
\end{tabular}
\end{table}


In this domain nine preference profiles are created in three sets of three each. Profiles 1, 2 and 3 will negotiate with one another, as will 4,5 and 6 and so on. The first three profiles are negotiating in a collaborative scenario, the second set is a moderate scenario and the last three profiles clash in a competitive scenario. All nine preference profiles are shown in Tables \ref{tab:prefprof1} through \ref{tab:prefprof9}.

\newpage


\begin{table}[H]
\centering
\caption{Preference profile 1}
\label{tab:prefprof1}
\begin{tabular}{|c|c|c|c|c|}
\hline
Power (hp) & Capacity & Mileage (km) & Fuel consumption (km/l) & Wheels \\
\hline
0.1 & 0.2 & 0.3 & 0.15 & 0.25 \\
\hline
2 & 3 & 3 & 2 & 3 \\
1 & 2 & 2 & 3 & 2 \\
  & 1 & 1 & 1 & 1 \\
  &   &   & 1 &   \\ 
\hline
\end{tabular}
\end{table}

\begin{table}[H]
\centering
\caption{Preference profile 2}
\label{tab:prefprof2}
\begin{tabular}{|c|c|c|c|c|}
\hline
Power (hp) & Capacity & Mileage (km) & Fuel consumption (km/l) & Wheels \\
\hline
0.15 & 0.2 & 0.35 & 0.1 & 0.2 \\
\hline
1 & 2 & 2 & 13 & 3 \\
1 & 2 & 2 & 12 & 3 \\
  & 1 & 1 & 1  & 1 \\
  &   &   & 10 &   \\ 
\hline
\end{tabular}
\end{table}

\begin{table}[H]
\centering
\caption{Preference profile 3}
\label{tab:prefprof3}
\begin{tabular}{|c|c|c|c|c|}
\hline
Power (hp) & Capacity & Mileage (km) & Fuel consumption (km/l) & Wheels \\
\hline
0.2 & 0.2 & 0.2 & 0.2 & 0.2 \\
\hline
2 & 3 & 3 & 4 & 3 \\
5 & 3 & 1 & 3 & 2 \\
  & 1 & 1 & 1 & 1 \\
  &   &   & 2 &   \\ 
\hline
\end{tabular}
\end{table}

\begin{table}[H]
\centering
\caption{Preference profile 4}
\label{tab:prefprof4}
\begin{tabular}{|c|c|c|c|c|}
\hline
Power (hp) & Capacity & Mileage (km) & Fuel consumption (km/l) & Wheels \\
\hline
0.4 & 0.15 & 0.2 & 0.1 & 0.15 \\
\hline
2 & 3 & 2 & 2 & 4 \\
1 & 2 & 5 & 3 & 3 \\
  & 1 & 4 & 1 & 1 \\
  &   &   & 1 &   \\ 
\hline
\end{tabular}
\end{table}

\begin{table}[H]
\centering
\caption{Preference profile 5}
\label{tab:prefprof5}
\begin{tabular}{|c|c|c|c|c|}
\hline
Power (hp) & Capacity & Mileage (km) & Fuel consumption (km/l) & Wheels \\
\hline
0.15 & 0.2 & 0.35 & 0.1 & 0.2 \\
\hline
1 & 2 & 4 & 13 & 4 \\
1 & 1 & 1 & 12 & 1 \\
  & 1 & 3 & 1  & 2 \\
  &   &   & 10 &   \\ 
\hline
\end{tabular}
\end{table}

\begin{table}[H]
\centering
\caption{Preference profile 6}
\label{tab:prefprof6}
\begin{tabular}{|c|c|c|c|c|}
\hline
Power (hp) & Capacity & Mileage (km) & Fuel consumption (km/l) & Wheels \\
\hline
0.4 & 0.15 & 0.15 & 0.15 & 0.15 \\
\hline
2 & 3 & 6 & 4 & 2 \\
5 & 3 & 4 & 3 & 3 \\
  & 1 & 1 & 1 & 6 \\
  &   &   & 2 &   \\ 
\hline
\end{tabular}
\end{table}

\begin{table}[H]
\centering
\caption{Preference profile 7}
\label{tab:prefprof7}
\begin{tabular}{|c|c|c|c|c|}
\hline
Power (hp) & Capacity & Mileage (km) & Fuel consumption (km/l) & Wheels \\
\hline
0.1 & 0.1 & 0.1 & 0.1 & 0.6 \\
\hline
2 & 6 & 6 & 5 & 8 \\
1 & 4 & 3 & 2 & 7 \\
  & 2 & 1 & 3 & 1 \\
  &   &   & 3 &   \\ 
\hline
\end{tabular}
\end{table}

\begin{table}[H]
\centering
\caption{Preference profile 8}
\label{tab:prefprof8}
\begin{tabular}{|c|c|c|c|c|}
\hline
Power (hp) & Capacity & Mileage (km) & Fuel consumption (km/l) & Wheels \\
\hline
0.35 & 0.05 & 0.05 & 0.05 & 0.5 \\
\hline
6 & 2 & 4 & 13 & 1 \\
1 & 3 & 3 & 10 & 6 \\
  & 4 & 1 & 1  & 7 \\
  &   &   & 10 &   \\ 
\hline
\end{tabular}
\end{table}

\begin{table}[H]
\centering
\caption{Preference profile 9}
\label{tab:prefprof9}
\begin{tabular}{|c|c|c|c|c|}
\hline
Power (hp) & Capacity & Mileage (km) & Fuel consumption (km/l) & Wheels \\
\hline
0.5 & 0.1 & 0.15 & 0.05 & 0.2 \\
\hline
1 & 5 & 1  & 1  & 12\\
4 & 1 & 10 & 20 & 2 \\
  & 3 & 25 & 18 & 8 \\
  &   &    & 17 &   \\ 
\hline
\end{tabular}
\end{table}




\newpage

\section*{Agent zelf uitleggen}
-De negotiating agent zelf.

\begin{itemize}
\item an explanation and motivation of all of the choices made in the design of negotiating agent.
\item should help the reader to understand the organization of the source code (important details should be commented on in the source code itself).This means that the main Java methods used by your agent should
be explained in the report itself.
\item a high-level description of the agent and its structure, including the main Java methods (mention these explicitly!) used in the negotiating agent that have been implemented in the source code.
\item an explanation of the negotiation strategy, decision function for accepting offers, any important preparatory steps, and heuristics that the agent uses to decide what to do next, including the factors that have been selected and their combination into these functions.
\end{itemize}

\section*{Tests we performed}

a section documenting the tests you performed to improve the negotiation strength of your agent.
You must include scores of various tests over multiple sessions that you performed while testing
your agent. Describe how you set up the testing situation and how you used the results to modify
your agent

\section*{Conclusions}

A conclusion in which you summarize your experience as a team with regards to building the
negotiating agent and discuss what extensions are required to use your agent in real-life negotiations to support (or even take over) negotiations performed by humans.

\newpage

\section*{Report requirements and checklist}

The report should include:\\
\begin{itemize}
\item \st{the group number}
\item an introduction to the assignment
\item a high-level description of the agent and its structure, including the main Java methods (mention
these explicitly!) used in the negotiating agent that have been implemented in the source code
\item an explanation of the negotiation strategy, decision function for accepting offers, any important
preparatory steps, and heuristics that the agent uses to decide what to do next, including the factors
that have been selected and their combination into these functions
\item a section documenting the tests you performed to improve the negotiation strength of your agent.
You must include scores of various tests over multiple sessions that you performed while testing
your agent. Describe how you set up the testing situation and how you used the results to modify
your agent
\item a conclusion in which you summarize your experience as a team with regards to building the
negotiating agent and discuss what extensions are required to use your agent in real-life negotiations
to support (or even take over) negotiations performed by humans.
\end{itemize}

\textbf{The final analysis report should involve an elaborate analysis of your agent's performance from different perspectives (e.g. individual utility gained, social welfare - the sum of utilities of all agents, optimality of the outcome, fairness etc.). NU NOG NIET NODIG}

\end{document}