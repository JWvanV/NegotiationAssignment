For the testing of our agent we actually employed an agile development technique. This allowed us to get fast results, and use them right away to implement changes.
\\\\
The real testing was done in the scenario described in paragraph~\ref{sec:scenario}. Our test case consisted of three parts for each run. We started the collaborative scenario (profiles 1, 2, and 3), the moderate scenario (profiles 4, 5, and 6), and the competitive scenario (profiles 7, 8, and 9), and compared the results to each previous run.
\\
Values from the output we considered were:
\begin{itemize}
\item Whether the graph looked like we expected;
\item Whether or not there was an agreement;
\item How soon the agreement was made;
\item How far the agreement product was from the nash line;
\item Distance to pareto;
\item Distance to Nash (which appeared to be different from the difference between the product and the nash line).
\end{itemize}

When this result was to our liking (i.e. when the solution was still pareto and the distance was closer to Nash), we used to previously implemented change, and went on to the next improvement. When it was not to our liking, or when there was an error, we tried to correct it, or we reviewed our thoughts about the usefulness of this feature.
\\\\
TODO: Hippe resultaten die we tegengekomen zijn